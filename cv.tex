%----------------------------------------------------------------------------------------
%	PACKAGES AND OTHER DOCUMENT CONFIGURATIONS
%----------------------------------------------------------------------------------------
%
\documentclass[11pt, a4paper, sans]{moderncv}      % possible options include font size ('10pt', '11pt' and '12pt'), paper size ('a4paper', 'letterpaper', 'a5paper', 'legalpaper', 'executivepaper' and 'landscape') and font family ('sans' and 'roman')
%
% moderncv themes
\moderncvstyle{casual}                             % style options are 'casual' (default), 'classic', 'banking', 'oldstyle' and 'fancy'
\moderncvcolor{blue}                               % color options 'black', 'blue' (default), 'burgundy', 'green', 'grey', 'orange', 'purple' and 'red'
%\renewcommand{\familydefault}{\sfdefault}         % to set the default font; use '\sfdefault' for the default sans serif font, '\rmdefault' for the default roman one, or any tex font name
%\nopagenumbers{}                                  % uncomment to suppress automatic page numbering for CVs longer than one page
% character encoding
%\usepackage[utf8]{inputenc}                       % if you are not using xelatex ou lualatex, replace by the encoding you are using
%\usepackage{CJKutf8}                              % if you need to use CJK to typeset your resume in Chinese, Japanese or Korean
%
% adjust the page margins
\usepackage[scale=0.75]{geometry}
%\setlength{\hintscolumnwidth}{3cm}                % if you want to change the width of the column with the dates
%\setlength{\makecvtitlenamewidth}{10cm}           % for the 'classic' style, if you want to force the width allocated to your name and avoid line breaks. be careful though, the length is normally calculated to avoid any overlap with your personal info; use this at your own typographical risks...
%
% personal data
\name{Scott C.}{Lowe}
%\title{Curriculum Vitae}
% optional, remove / comment the line if not wanted; the "postcode city" and "country" arguments can be omitted or provided empty
\address{University of Edinburgh, Informatics Forum, 10 Crichton Street}{Edinburgh, EH8~9AB}{U.K.}
% optional, remove / comment the line if not wanted; the optional "type" of the phone can be "mobile" (default), "fixed" or "fax"
%\phone[mobile]{+1~(234)~567~890}
%\phone[fixed]{+2~(345)~678~901}
%\phone[fax]{+3~(456)~789~012}
\email{scottclowe@gmail.com}
%\homepage{www.johndoe.com}
\social[linkedin]{scottclowe}
%\social[twitter]{scottclowe}
\social[github]{scottclowe}
\social[kaggle]{scottclowe}
\extrainfo{Nationality: British}
%\photo[64pt][0.4pt]{picture}
% optional, remove / comment the line if not wanted; '64pt' is the height the picture must be resized to, 0.4pt is the thickness of the frame around it (put it to 0pt for no frame) and 'picture' is the name of the picture file
%\quote{Some quote}
%
% bibliography adjustements (only useful if you make citations in your resume, or print a list of publications using BibTeX)
%   to show numerical labels in the bibliography (default is to show no labels)
\makeatletter\renewcommand*{\bibliographyitemlabel}{\@biblabel{\arabic{enumiv}}}\makeatother
%   to redefine the bibliography heading string ("Publications")
%\renewcommand{\refname}{Articles}
%
% bibliography with mutiple entries
%\usepackage{multibib}
%\newcites{book,misc}{{Books},{Others}}
%----------------------------------------------------------------------------------------
%
\usepackage{multirow}
%
%\usepackage[usenames,dvipsnames]{xcolor}
%\newcommand\colorhref[3][MidnightBlue]{\href{#2}{\color{#1}#3}}
\newcommand\colorhref[2]{\href{#1}{\textcolor[HTML]{006794}{#2}}}
%
\begin{document}
%
%------------------------------------------------
\makecvtitle

\section{Current position}
% Your current or previous employment position

\cventry{2011--2016}{Ph.D. candidate}{Institute for Adaptive and Neural Computation, School of Informatics}{University of Edinburgh}{}{Project: Analysis of experimental data from multi-electrode recordings in the visual cortex. Supervisors: Mark van Rossum, Stefano Panzeri and Alex Thiele.}
%\cvitem{description}{Short thesis abstract}


\section{Education}
% arguments 3 to 6 can be left empty
%\cventry{year--year}{Degree}{Institution}{City}{\textit{Grade}}{Description}

%\item \years{2012--present}{PhD candidate} in Neuroinformatics and Computational Neuroscience, University of Edinburgh

\cventry{2011--2012}{MSc with Distinction, in Neuroinformatics by Research}{University of Edinburgh}{Edinburgh, UK}{\textit{average 75.3\%%
% --- for comparison, a Distinction at $\ge70\%$ is approximately equivalent to $\ge3.75/4$ US GPA
}}{%
%\begin{tabular}{lp{11.5cm}}
% & Neural Computation, Principles of Neuroscience, Probabilistic Modelling and Reasoning, Neural Information Processing, Computational Neuroscience of Vision, Neuroinformatics Research, Research Thesis (Neuroinformatics; Topic: \emph{An information theoretic analysis of perceptual learning data from macaque V1 and V4})\\
%\end{tabular}
Thesis: \emph{An information theoretic analysis of perceptual learning data from macaque V1 and V4},  Supervisors: Alex Thiele and Stefano Panzeri.
}

\cventry{2007--2011}{MSci with First Class Honours, in Natural Sciences (Mathematics and Physics)}{Durham University}{Durham, UK}{\textit{average 73.4\%%
% --- for comparison, a First Class at $\ge70\%$ is approximately equivalent to $\ge3.75/4$ US GPA
}}{%
%\begin{tabular}{lp{11.5cm}}
%Y1:&~Core Maths A, Core Maths B1, Foundations of Physics 1, Discovery Skills in Physics\\
%Y2:&~Complex Analysis, Linear Algebra, Analysis in Many Variables, Foundations of Physics 2, Thermal and Condensed Matter Physics\\
%Y3:&~Differential Geometry, Dynamical Systems, Continuum Mechanics, Foundations of Physics 3, Theoretical Physics\\
%Y4:&~Elliptic Functions, Mathematical Finance, Partial Differential Equations, Solitons, Project (Mathematics; Topic: \emph{Artifical Neural Networks})\\
%\end{tabular}
Thesis: \emph{On Artifical Neural Networks},  Supervisor: Ian Jermyn.
}


%\section{Master thesis}
%\cvitem{title}{\emph{An information theoretic analysis of perceptual learning data from macaque V1 and V4}, Master of Science by Research thesis, University of Edinburgh.}
%\cvitem{year}{2012}
%\cvitem{supervisors}{Alex Thiele and Stefano Panzeri}
%%\cvitem{description}{Short thesis abstract}
%
%
%\section{Master thesis}
%\cvitem{title}{\emph{On Artifical Neural Networks}, Master of Natural Sciences (Mathematics) dissertation, Durham University}
%\cvitem{year}{2011}
%\cvitem{supervisors}{Ian Jermyn}
%%\cvitem{description}{Short thesis abstract}

\section{Online Courses}

\cventry{Feb 2016}{Machine Learning}{Stanford University}{via Coursera}{\colorhref{https://www.coursera.org/account/accomplishments/records/C3BHEHRLC4HT}{\textit{Grade: 100\%}}}{}


\section{Experience}

%\cventry{year--year}{Job title}{Employer}{City}{}{General description no longer than 1--2 lines.}

\cventry{2015}{Technical Research Assistant}{Rochefort Lab, Centre for Integrative Physiology}{University of Edinburgh}{}{Development of tools for analysis of calcium imaging data from mouse primary visual cortex.
%\url{https://github.com/rochefort-lab}.
}

\cventry{2010}{Web Technician}{FleXtel Ltd}{Sandbach, UK}{}{Programming in PHP for telecoms company. Designed and coded new website selling isolated consumer product. Developed market-leading algorithms to price telephone numbers patterns based on memorability of both numeric patterns and alphadial patterns \url{http://www.flextel.com/numbers/}.
}

\cventry{Sept, 2009}{Physics Studentship}{University of Durham}{Durham}{}{Programming in MATLAB to simulate Rydberg atoms and their interactions.
}

%------------------------------------------------
\section{Open Source Projects}

\cventry{2013--2015}{MATLAB Schemer}{A colour scheme manager for MATLAB}{available on \colorhref{https://github.com/scottclowe/matlab-schemer}{GitHub}, and through \colorhref{http://mathworks.com/matlabcentral/fileexchange/53862-matlab-schemer}{MATLAB FileExchange}}{}{}

\cventry{2013}{Colorlab}{Perceptually uniform colormap generation}{available on \colorhref{https://github.com/scottclowe/colorlab}{GitHub}}{}{}

%----------------------------------------------------------------------------------------
%	GRANTS, HONORS AND AWARDS SECTION
%----------------------------------------------------------------------------------------

\section{Grants, honors \& awards}

\cvitem{2015}{
Placed 57th out of 1049 in the \colorhref{https://www.kaggle.com/c/datasciencebowl}{National Data Science Bowl} plankton species classification challenge, hosted by Kaggle.
}

\cvitem{2014}{
Placed 16th out of 504 in the \colorhref{https://www.kaggle.com/c/seizure-prediction}{American Epilepsy Society Seizure Prediction Challenge}, hosted by Kaggle.
}

\cvitem{2013}{
Winner of ``Most Viable Business Idea'' award, \colorhref{http://www.meetup.com/Amazon-UK-Hackathon-Group/events/108265382/}{Amazon Scotland Hackathon 2013}.
}

\cvitem{2011}{
Awarded a 4-year scholarship by the University of Edinburgh School of Informatics Doctoral Training Centre in Neuroinformatics, with funding from grants EP/F500385/1 and  BB/F529254/1 from the UK Engineering and Physical Sciences Research Council (EPSRC), UK Biotechnology and Biological Sciences Research Council (BBSRC), and the UK Medical Research Council (MRC).
}

%----------------------------------------------------------------------------------------

\section{Computer skills}

\cvitem{Programming}{Python, MATLAB}
\cvitem{Web dev}{PHP, Javascript, AJAX, HTML, XHTML}
\cvitem{Database}{SQL, MySQL}
\cvitem{Cloud computing}{Amazon EC2}

%----------------------------------------------------------------------------------------
%	PUBLICATIONS AND TALKS SECTION
%----------------------------------------------------------------------------------------

\section{Publications}

% % Publications from a BibTeX file without multibib
% %  for numerical labels: \renewcommand{\bibliographyitemlabel}{\@biblabel{\arabic{enumiv}}}% CONSIDER MERGING WITH PREAMBLE PART
% %  to redefine the heading string ("Publications"): \renewcommand{\refname}{Articles}
% \nocite{*}
% \bibliographystyle{plain}
% \bibliography{publications}    

\subsection{Journal articles}

\cvitem{Sept, 2015}{
Michel Besserve, Scott C. Lowe, Nikos, K. Logothetis, Bernhard Sch\"{o}lkopf, Stefano Panzeri (2015, September), ``Shifts of gamma phase across primary visual cortical sites reflect dynamic stimulus modulated information transfer'', \emph{PLOS Biology}. \colorhref{https://doi.org/10.1371/journal.pbio.1002257}{DOI: 10.1371/journal.pbio.1002257}.
}

\cvitem{---}{
Janelle Pakan, Scott C. Lowe, Evelyn Dylda, Sander Keemink, Christopher Coutts, Nathalie L. Rochefort, (in review), ``Behavioural state modulation of inhibitory activity is context-dependent in mouse V1''.
}

\cvitem{---}{
Scott C. Lowe, Daniel Zaldivar, Yusuke Murayama, Mark C. W. van Rossum, Nikos K. Logothetis, Stefano Panzeri (to be submitted), ``Lamina and Frequency Distribution of Information in Primary Visual Cortex''.
}

\cvitem{---}{
Sander W. Keemink*, Scott C. Lowe*, Janelle M. P. Pakan, Mark C. W. van Rossum, Nathalie L. Rochefort (in preparation), ``FISSA: Fast 2-photon signal extraction and separation''.
}

\cvitem{---}{
Scott C. Lowe, Xing Chen, Alex Thiele, Mark C. W. van Rossum, Stefano Panzeri (in preparation), ``Changes in V1 and V4 encoding of visual contrast during perceptual learning''.
}

\cvitem{---}{
Scott C. Lowe, Finlay Maguire, Gavin Gray (in preparation), ``Predicting the onset of epileptic seizures from intracranial-EEG: which features are most useful''.
}


\subsection{Talks}

\cvitem{May, 2015}{
Scott C. Lowe (2015, May),``What does LFP encode?''.
{Presented at the \emph{CINPLA Workshop: ``Inferring network activity from LFPs''}, University of Oslo, Oslo, Norway.}
}

%\cvitem{}{
% Scott C. Lowe (2014, January),``Independent channels of information in visual cortical Local Field
% Potentials''.
% {Presented at the \emph{Institute for Adaptive and Neural Computation Workshop, Jan 14, 2014}, Edinburgh.}
%}

\subsection{Poster Presentations}
%
% Janelle Pakan, Scott C. Lowe, ..., Nathalie Rochefort. EVCM
%
% Sander Keemink, Janelle Pakan, Scott C. Lowe, ..., Nathalie Rochefort. EVCM

\cvitem{Apr, 2015}{
Scott C. Lowe, \textit{et al.} (2015, April), ``Cortical dynamics across V1 laminae generate independent frequency channels encoding visual information''.
Presented at the \emph{BNA2015: Festival of Neuroscience}, Edinburgh, UK. Poster Reference: \colorhref{http://www.bna.org.uk/static/docs/BNA2015/BNA2015-Abstract-Book.pdf}{P2-C-029}.
}

\cvitem{Nov, 2014}{
Scott C. Lowe, \textit{et al.} (2014, November), ``Different cortical layers in V1 encode different visual information in different frequency bands''.
Presented at the \emph{2014 Meeting of the Society for Neuroscience}, Washington DC, USA. Program No. \colorhref{http://www.abstractsonline.com/Plan/ViewAbstract.aspx?sKey=a018159d-9116-492d-b234-b4c53acd5260&cKey=e909b23c-40c3-4371-8816-345dfe3e6c3c&mKey=54c85d94-6d69-4b09-afaa-502c0e680ca7}{532.19}.
}

\cvitem{July, 2014}{
Scott C. Lowe, \textit{et al.} (2014, July), ``Quantification of the Laminar and Frequency Structure of Information in Primary Visual Cortex''.
Presented at the \emph{9th FENS Forum of Neuroscience}, Milan, Italy. Abstract number \colorhref{http://fens2014.meetingxpert.net/FENS_427/poster_102139/program.aspx}{FENS-2860}.
}

\cvitem{July, 2014}{
Scott C. Lowe, \textit{et al.} (2014, July), ``Quantification of the Laminar and Frequency Structure of Information in Primary Visual Cortex''.
Presented at the \emph{AREADNE 2014 session}, Santorini, Greece.
}

\cvitem{Nov, 2013}{
Scott C. Lowe, \textit{et al.} (2013, November), ``Decoding spiking activity in V4, but not V1, correlates with behaviour in perceptual learning''.
Presented at the \emph{2013 Meeting of the Society for Neuroscience}, San Diego, USA. Program No. \colorhref{http://www.abstractsonline.com/Plan/ViewAbstract.aspx?sKey=a69ca081-1031-4c5e-917c-b57b7b7255cf&cKey=5e73e1d9-7177-4207-9890-3efb2a57985b&mKey=8d2a5bec-4825-4cd6-9439-b42bb151d1cf}{555.11.}
}

\cvitem{July, 2013}{
Scott C. Lowe, \textit{et al.} (2013, July), ``Decoding spiking activity in V4, but not V1, correlates with behavioural performance in perceptual learning task''.
Presented at the \emph{Twenty Second Annual Computational Neuroscience Meeting: CNS*2013}, Paris, France.
\emph{BMC Neuroscience} 2013, \textbf{14}(Suppl 1):P385  \colorhref{http://dx.doi.org/10.1186/1471-2202-14-S1-P385}{doi:10.1186/1471-2202-14-S1-P385}.
}

%----------------------------------------------------------------------------------------
% REFERENCES
%----------------------------------------------------------------------------------------

\section{References}
\cvlistitem{Dr. Mark van Rossum,\\
School of Informatics,\\
University of Edinburgh,\\
Edinburgh, EH8 9AB, UK\\
\colorhref{mailto:mvanross@inf.ed.ac.uk}{mvanross@inf.ed.ac.uk}\\
}
\cvlistitem{Prof. Stefano Panzeri,\\
Center for Neuroscience and Cognitive Systems,\\
Istituto Italiano di Technologia,\\
Bettini 31, Rovereto (Tn), Italy\\
\colorhref{mailto:stefano.panzeri@iit.it}{stefano.panzeri@iit.it}\\
}
\cvlistitem{Dr. Nathalie Rochefort,\\
Centre for Integrative Physiology,\\
University of Edinburgh,\\
Edinburgh, EH8 9XD, UK\\
\colorhref{mailto:n.rochefort@ed.ac.uk}{n.rochefort@ed.ac.uk}\\
}
\cvlistitem{Additional references available on request.}

%----------------------------------------------------------------------------------------
%
\end{document}
